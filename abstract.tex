%
We present an analysis of the seismicity of northern Iran in the period between 2005 and 2015 using a recently introduced method based on concepts of graph theory. The method relies on the inter-event visibility defined in terms of a connectivity degree parameter, $k$, which is correlated with the earthquake magnitude, $M$. Previous studies show that the slope of the $k$-$M$ correlation also observes a relationship with the $b$-value from the Gutenber-Richter law. Thus rendering the graph analysis useful to examine the seismicity of a region. In addition, other topological properties of the network and dynamic properties of the magnitude-time series exhibit characteristics that seem to be associated with the occurrence of earthquakes, offering the possibility of correlating seismicity parameters with time. We apply this approach to the case of the seismicity of northern Iran, using a declustered seismic catalogs for the tectonic seismic regions of Azerbaijan, Alborz, and Kopeh Dagh. We use results draw for this region with the visibility graph approach in combination with results from other similar studies to further improve the universal correlation between $k$-$M$ and $b$-values and show that the visibility graph approach can be considered as a valid alternative for analyzing regional seismicity properties and earthquake sequences.

% --------------------------------------------------------------------------------------------------

% NAEEM'S ABSTRACT
% ----------------

% The seismicity of northern Iran between 2005 and 2015 is investigated by means of the visibility graph (VG) method. For purposes of this study, northern Iran is divided into three tectonic seismic regions: Azerbaijan, Alborz, and Kopeh Dagh. Using the declustered catalog, we studied the VG properties of the magnitude time series. The results show a relationship between the $k-M$ slope and the $b-value$ of the Gutenberg-Richter law. Topological properties (i.e.  $< T_c >$, $k-M$ slope) of the network and dynamic properties of magnitude time series (i.e. $b-value$) significantly decrease before large earthquakes. Combining the results of this study with three similar previous studies improves the linear correlation coefficient factor between $k-M$ slope and $b-value$, and empowers the idea of a universal relationship between $b-value$ and $k-M$ slope. The behaviors of the VG's properties are similar to previous studies and could be considered an alternative method for analyzing earthquake sequences. 
