%
The seismicity of northern Iran between 2005 and 2015 is investigated by means of the visibility graph (VG) method. For purposes of this study, northern Iran is divided into three tectonic seismic regions: Azerbaijan, Alborz, and Kopeh Dagh. Using the declustered catalog, we studied the VG properties of the magnitude time series. The results show a relationship between the $k-M$ slope and the $b-value$ of the Gutenberg-Richter law. Topological properties (i.e.  $< T_c >$, $k-M$ slope) of the network and dynamic properties of magnitude time series (i.e. $b-value$) significantly decrease before large earthquakes. Combining the results of this study with three similar previous studies improves the linear correlation coefficient factor between $k-M$ slope and $b-value$, and empowers the idea of a universal relationship between $b-value$ and $k-M$ slope. The behaviors of the VG's properties are similar to previous studies and could be considered an alternative method for analyzing earthquake sequences. 
