%
We present an analysis of the seismicity of northern Iran in the period between 2005 and \myrevision{2016} using a recently introduced method based on concepts of graph theory. The method relies on the inter-event visibility defined in terms of a connectivity degree parameter, $k$, which is correlated with the earthquake magnitude, $M$. Previous studies show that the slope of the $k$-$M$ correlation also observes a relationship with the $b$-value from the Gutenberg-Richter law, thus rendering the graph analysis useful to examine the seismicity of a region. In addition, other topological properties of the network and dynamic properties of the magnitude-time series exhibit characteristics that seem to be associated with the occurrence of earthquakes, offering the possibility of correlating seismicity parameters with time. We apply this approach to the case of the seismicity of northern Iran, using an earthquake catalog for the tectonic seismic regions of Azerbaijan, Alborz, and Kopeh Dagh. We use results drawn for this region with the visibility graph approach in combination with results from other similar studies to further improve the universal relationship between the slope of $k$-$M$ and $b$-value, and show that the visibility graph approach can be considered as a valid alternative for analyzing regional seismicity properties and earthquake sequences.
