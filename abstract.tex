\begin{abstract}
The seismicity of northern Iran between 2005 and 2015 is investigated by means of visibility graph (VG) method. According to the seismicity, in this study the northern Iran is divided into three tectonic seismic regions including Azerbaijan, Alborz, and Kopeh Dagh.  The aftershocks and foreshocks are removed from the catalog and the magnitude time series are generated for each tectonic seismic zone. Using declustered catalog, we studied the VG properties of the magnitude time series. The results show a relationship between the $k-M$ slope and the $b-value$ of the Gutenberg-Richter law. Topological properties (i.e. $< T_c >$, $k-M$ slope) of network and  dynamic properties of magnitude time series (i.e. $b-value$) significantly decrease before large earthquakes. Cumulating the results of this study into two similar previous studies, improves the linear correlation coefficient factor and empower the idea of universal relationship between $b-value$ and $k-M$ slope. The behavior of  the VG's properties are similar to previous studies and could be considered an alternative method to analyze the earthquake sequence. 
\end{abstract}
