
\section{Introduction}

In a relatively recent study, \citet{Lacasa2008} introduced a simple and computationally inexpensive approach to convert time-series into a particular kind of mathematical graphs known as visibility graphs. In this approach, the connection between the nodes in a network---here understood as events in a time series---is based on the visibility of each node or event with respect to other previous and later events in a time sequence. Over the last decade, this method has been found to be applicable to multiple fields, including economics, medicine, and geophysics \citep[e.g.,][]{Yang_2009_PA, Elsner_2009_GRL, Telesca2012_CSF, Wang2012, Long2013}.

For the particular case of seismological applications, previous studies have proposed to consider the nodes in the graph to be the earthquakes in a given time series of events, and analyzed the properties of the graph with respect to traditional seismicity parameters. The analysis of seismic sequences through the visibility graph approach for various tectonic seismic regions has proven to be a valid alternative to studying magnitude time series. \citet{Telesca2012}, for instance, studied the seismicity of Italy between the years 2005 and 2010 using the visibility graph method. Applying different threshold magnitudes to construct the graph and observing the collapsing effect of all distribution degrees, they observed that the properties of the visibility graph seemed to depend only on the magnitude values, and not on the threshold magnitude used in the analysis. 

Subsequently, \citet{Telesca2013} studied the seismicity of the Mexican subduction zone through the visibility graph approach and found that the properties of the graph were correlated to the seismic $b$-value in the Gutenber-Richter law \citep{Gutenberg1944}. In particular, \citet{Telesca2013} extracted the characteristics of the visibility graph for five different tectonic seismic regions in the subduction zone and found that the slope in the linear relationship between earthquake magnitude, $M$, and the inter-event visibility in the graph defined as the connectivity degree parameter, $k$, was correlated with the $b$-value.

In a similar article, \citet{Telesca2014} studied the seismicity of the 2002--2011 Pannonian region using the visibility graph method for two sub-catalogs of shallow and deep earthquakes. They extracted the visibility graph characteristics for each group of events and confirmed that there was a close correlation between the Gutenberg-Richter $b$-value and the slop of the $k$-$M$ relationship obtained from the visibility graph. According to \citet{Telesca2014}, the high linear correlation coefficient value (close to 1.0) between the $b$-value and the $k$-$M$ slope indicates that this relationship exhibits a nearly universal character. This observation was reinforced experimentally by \citet{Telesca2014-pone}, who investigated the behavior of a mechanical stick-slip system with different asperities (sandpaper of different grades). After collecting data for different experiments emulating young and mature faults, \citet{Telesca2014-pone} found again that the $b$-value was linearly correlated with the $k$-$M$ slope derived from the visibility graph of the experimental events.

More recently, \citet{Telesca2016} used the visibility graph method to gain insight about the timespan between earthquakes. The authors defined the parameter <$T_c$> as the window mean interval connectivity time, which provides information about the average of all the time intervals between visible events. Applying this concept to the 2003--2012 earthquake sequence in Kachchh, western India, \citet{Telesca2016} found that the variation of <$T_c$> in time exhibited a plausible relationship with the occurrence of earthquakes. In particular, they observed that the value of <$T_c$> decreases significantly before the largest shock in the Kachchh catalog.

In this article we use the visibility graph method to analyze the seismicity of northern Iran from 2005 to 2015. We focus our analysis on these years because in a previous study we observed that the latest decade of seismic records in the region offered the most complete catalog of events \citep[e.g.][]{Khoshnevis2016}. Upon a brief review of the basic concepts of the method and its applications, we describe the seismicity of northern Iran considering three dominant seismic regions: Azerbaijan, Alborz, and Kopeh Dagh. These are regional areas of considerable ground shaking activity, with average seismic rates varying between about 0.17 and 3.25 events per year \citep[e.g.,][]{Nemati2015}. The considerable amount of earthquakes registered in a 10-year period facilitates the analysis done with the visibility graph approach within what would otherwise be considered a relatively short period of time, and serves as a good case study to test the capabilities of the method. We first summarize the dataset, including a description of the seismic catalog and its completeness based on a recent work on the seismicity of the region \citep{Khoshnevis2016}. We then present results obtained for the region regarding the relationship between the the graph's $k$-$M$ slope and the seismicity $b$-values for the three seismic areas, analyze the sensitivity of the catalogs to the number of events and the threshold magnitude, and present results for the variation of <$T_c$> through time.
