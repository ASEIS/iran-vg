
\section{Introduction}

In a relatively recent study, \citet{Lacasa2008} introduced a simple and computationally inexpensive approach to convert time-series into mathematical graphs. The method, known as the visibility algorithm, represents the connection between nodes in a network based on the visibility of each node in a time sequence with respect to previous and later nodes in the sequence. Over the last decade, this method has been found to be applicable to multiple fields, including economics and geophysics \citep[e.g.,][]{Telesca2012_CSF, Long2013, Wang2012}. 

For the particular case of seismological applications, previous studies have proposed the nodes in the graph to be considered as the earthquakes in a given time series of events, and analyzed the properties of the graph with respect to traditional seismicity parameters. The analysis of seismic sequences through the visibility graph approach for various tectonic seismic regions has proven to be a valid alternative to studying magnitude time series. \citet{Telesca2012}, for instance, studied the seismicity of Italy between the years 2005 and 2010 using the visibility graph method. Applying different threshold magnitudes to construct the graph and observing the collapsing effect of all distribution degrees, they observed that the properties of the visibility graph seemed to depend only on the magnitude values, and not on the threshold magnitude used in the analysis. 

Subsequently, \citet{Telesca2013} studied the seismicity of the Mexican subduction zone through the visibility graph approach and found that the properties of the graph were correlated to the seismic $b$-value in the Gutenber-Richter law \citep{Gutenberg1944}. In particular, \citet{Telesca2013} extracted the characteristics of the visibility graph for five different tectonic seismic regions in the subduction zone and found that the slope in the linear relationship between earthquake magnitude, $M$, and the inter-event visibility in the graph defined as the connectivity degree parameter, $k$, was correlated with the $b$-value.

In a similar article, \citet{Telesca2014} studied the seismicity of the 2002--2011 Pannonian seismicity using the visibility graph method for two sub-catalogs of shallow and deep earthquakes. They extracted the visibility graph characteristics for each group of events and confirmed that there was a close correlation between the Gutenberg-Richter $b$-value and the slop of the $k$-$M$ relationship obtained from the visibility graph. According to \citet{Telesca2014}, the high linear correlation coefficient value (close to 1.0) between the $b$-value and the $k$-$M$ slope indicates that this relationship exhibits a nearly universal character. This observation was reinforced experimentally by \citet{Telesca2014-pone}, who investigated the behavior of a mechanical stick-slip system with different asperities (sandpaper of different grades). After collecting data for different experiments emulating young and mature faults, \citet{Telesca2014-pone} found again that the $b$-value was linearly correlated with the $k$-$M$ slope derived from the visibility graph of the experimental events.

More recently, \citet{Telesca2016} used the visibility graph method to gain insight about the timespan between earthquakes. The authors defined the parameter window mean interval connectivity time <$T_c$>, which provides information about the average of all the time intervals between visible events. Applying this concept to the 2003--2012 earthquake sequence in Kachchh, western India, \citet{Telesca2016} found that the variation of <$T_c$> in time exhibited a plausible relationship with the occurrence of earthquakes. In particular, they observed that the value of <$T_c$> decreases significantly before the largest shock in the Kachchh catalog.

In this article we use the visibility graph method to analyze the seismicity of northern Iran for the 2005--2015 period. We focus our analysis the last decade period because previous studies on the analysis of seismic catalogs for the region have shown that this period has the most complete datasets, especially for the case of small earthquakes with $M<4$ \citep[e.g.][]{Khoshnevis2016}. We analyze this region considering three dominant seismic areas: Azerbaijan, Alborz, and Kopeh Dagh.

...

% NAEEM'S INTRODUCTION
% --------------------

% This article presents the application of VG method on seismicity of northern Iran using the recorded earthquake during the last 10 years (2005-15). Recent studies show that this period (2005-15) has the most complete catalog especially for small earthquakes $(M < 3)$ \citep[e.g.][]{Khoshnevis2016}.  \citet{Lacasa2008} introduced a simple and fast computational method, known as visibility algorithm, to convert time series into graphs. Basically, the visibility graph presents the connection between nodes in a network. This concept has applications in many fields including geophysical studies\citep[e.g.,][]{Telesca2012_CSF,Long2013,Wang2012}.

% The analysis of seismic sequence through visibility graph for various tectonic seismic regions revealed an alternative approach to studying the magnitude time series. \citet{Telesca2012} studied the seismic sequence of Italy between the years 2005 and 2010 using the visibility graph method. Applying different threshold magnitudes and observing the collapsing effect of all distribution degrees, they argued that in analyzing the magnitude time series, VG seems to depend on only the values of the magnitude and being independent of the threshold magnitude. 

% In other research, \citet{Telesca2013} studied the seismicity of the Mexican subduction zone through the visibility graph approach. They extracted characteristics of the visibility graph for five different tectonic seismic regions.  According to their study, a visibility graph could distinguish the region with different seismicity characteristics, resulting from its different tectonic processes governing the area. Their study investigated the relationship between the Gutenberg-Richter parameters and the $k-M$ slope, which is the slope of the regression line fitting the magnitude of each event ($M$) and its connectivity degree ($k$). 

% In a similar article, \citet{Telesca2014} studied the seismicity of the 2002-11 Pannonian seismicity using visibility graph. They classified the seismic catalog into shallow and deep earthquakes, and extracted the visibility graph characteristics for each class. According to their study, a close relationship exists between the Gutenberg-Richter $b-value$ and the $k-M$ slope of the visibility graph. The high linear correlation coefficient value (close to 1) of $b-value$ and $k-M$  slope suggests the universal character of the relationship between these parameters \citep{Telesca2014}.  Using the synthetic seismicity generated by a simple stick-slip system with asperities, \citet{Telesca2014-pone} investigated the relationship between the $b-value$ of the Gutenberg-Richter law and the slope of the $k-M$ plot as obtained by the VG method. They build up a frictional system as a proxy of geological faults under tectonic stress and defined different asperities by means of modifying the sandpaper grade. They collected data after the first run ($R_1$) and the fifth run ($R_5$) mimicking the younger and mature faults, respectively. They calculated the magnitude as a normalized displacement in range of $1-10$. According to their study the $b-value$ and the $k-M$ slope are linearly correlated to each other with a high correlation coefficient.

% VG method also is applied through considering the timespan between earthquakes. \citet{Telesca2016} defined $<T_c>$ as the parameter window mean interval connectivity time, as an indication of the mean linkage time between earthquakes, in studying the 2003-12 earthquake sequence in the Kachchh region of western India. They found that $<T_C>$ changes through time, indicating that the topological properties of the earthquake network are not stationary. In addition, $<T_c>$ significantly decrees before the largest shock of the catalog. Analyzing the Gutenberg-Richter values of  the magnitude time series of an earthquake sequence revealed more opportunities to study the magnitude time series by using the features of networks. 

% In this study, we use visibility graph analysis to investigate the earthquake sequence of northern Iran. We divided northern Iran into three tectonic seismic regions, namely Azerbaijan (Az), Alborz Mountain Range (Al), and Kopeh Dagh (Ko).  Recently, northern Iran has been studied in detail from different seismological points of view.  \citet{Nemati2015} studied the most recent 200 years of seismicity in northern Iran. The frequency of shocks varied widely, from one main shock per 6 years (0.17 event/ year) for the Azerbaijan region to 13 earthquakes per 4 years (3.25 event/year) for the Kopeh-Dagh region \citep{Nemati2015}. Using the smoothed seismicity method, \citet{Khoshnevis2016} studied the seismic hazard of the three northern tectonic seismic regions. Having a considerable number of recorded earthquakes, extensive seismic history, and being in a unique tectonic seismic region, northern Iran is a good platform from which to analyze the visibility graph method's capabilities for studying the magnitude time series. Significant aspect of our work includes the compassion of earthquake catalog, conversion of the earthquake magnitude and computation of the completeness magnitude for each tectonic seismic zone. We present the relationship between  $k$-$M$  slope and $b$-value in these three tectonic seismic regions. We analyze the sensitivity of the catalogs to the number of events and the threshold magnitude. We also investigate the variation of  $<T_c>$  value through time. 
