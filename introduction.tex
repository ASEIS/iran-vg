\section{Introduction}
\noindent
This article presents the application of VG method on seismicity of northern Iran using the recorded earthquake during the last 10 years (2005-15). Recent studies show that this period (2005-15) has the most complete catalog especially for small earthquakes $(M < 3)$ \citep[e.g.][]{Khoshnevis2016}.  \citet{Lacasa2008} introduced a simple and fast computational method, known as visibility algorithm, to convert time series into graphs. Basically, the visibility graph presents the connection between nodes in a network. This concept has applications in many fields including geophysical studies\citep[e.g.,][]{Telesca2012_CSF,Long2013,Wang2012}. \\
\noindent
The analysis of seismic sequence through visibility graph for various tectonic seismic regions revealed an alternative approach to studying the magnitude time series.  \citet{Telesca2012} studied the seismic sequence of Italy between the years 2005 and 2010 using the visibility graph method. Applying different threshold magnitudes and observing the collapsing effect of all distribution degrees, they argued that in analyzing the magnitude time series, VG seems to depend on only the values of the magnitude and being independent of the threshold magnitude. 
\noindent
In other research, \citet{Telesca2013} studied the seismicity of the Mexican subduction zone through the visibility graph approach. They extracted characteristics of the visibility graph for five different tectonic seismic regions.  According to their study, a visibility graph could distinguish the region with different seismicity characteristics, resulting from its different tectonic processes governing the area. Their study investigated the relationship between the Gutenberg-Richter parameters and the $k-M$ slope, which is the slope of the regression line fitting the magnitude of each event  ($M$) and its connectivity degree  ($k$). 
\noindent
In a similar article, \citet{Telesca2014} studied the seismicity of the 2002-11 Pannonian seismicity using visibility graph. They classified the seismic catalog into shallow and deep earthquakes, and extracted the visibility graph characteristics for each class. According to their study, a close relationship exists between the Gutenberg-Richter $b-value$ and the $k-M$ slope of the visibility graph. The high linear correlation coefficient value (close to 1) of $b-value$ and $k-M$  slope suggests the universal character of the relationship between these parameters \citep{Telesca2014}.  Using the synthetic seismicity generated by a simple stick-slip system with asperities, \citet{Telesca2014-pone} investigated the relationship between the $b-value$ of the Gutenberg-Richter law and the slope of the $k-M$ plot as obtained by the VG method. They build up a frictional system as a proxy of geological faults under tectonic stress and defined different asperities by means of modifying the sandpaper grade. They collected data after the first run ($R_1$) and the fifth run ($R_5$) mimicking the younger and mature faults, respectively. They calculated the magnitude as a normalized displacement in range of $1-10$. According to their study the $b-value$ and the $k-M$ slope are linearly correlated to each other with a high correlation coefficient.\\
\noindent
VG method also is applied through considering the timespan between earthquakes. \citet{Telesca2016}   defined    $<T_c>$  as the parameter window mean interval connectivity time ,  as an indication of the mean linkage time between earthquakes, in studying the 2003-12 earthquake sequence in the Kachchh region of western India. They found that  $<T_C>$  changes through time, indicating that the topological properties of the earthquake network are not stationary. In addition,  $<T_c>$  significantly decrees before the largest shock of the catalog. Analyzing the Gutenberg-Richter values of  the magnitude time series of an earthquake sequence revealed more opportunities to study the magnitude time series by using the features of networks. 
\noindent
In this study, we use visibility graph analysis to investigate the earthquake sequence of northern Iran. We divided northern Iran into three tectonic seismic regions, namely Azerbaijan (Az), Alborz Mountain Range (Al), and Kopeh Dagh (Ko).  Recently, northern Iran has been studied in detail from different seismological points of view.  \citet{Nemati2015} studied the most recent 200 years of seismicity in northern Iran. The frequency of shocks varied widely, from one main shock per 6 years (0.17 event/ year) for the Azerbaijan region to 13 earthquakes per 4 years (3.25 event/year) for the Kopeh-Dagh region \citep{Nemati2015}. Using the smoothed seismicity method, \citet{Khoshnevis2016} studied the seismic hazard of the three northern tectonic seismic regions. Having a considerable number of recorded earthquakes, extensive seismic history, and being in a unique tectonic seismic region, northern Iran is a good platform from which to analyze the visibility graph method's capabilities for studying the magnitude time series. Significant aspect of our work includes the compassion of earthquake catalog, conversion of the earthquake magnitude and computation of the completeness magnitude for each tectonic seismic zone. We present the relationship between  $k-M$  slope and $b-value$  in these three tectonic seismic regions. We analyze the sensitivity of the catalogs to the number of events and the threshold magnitude. We also investigate the variation of  $<T_c>$  value through time. 



% Original text
%\section{Introduction}
%\noindent
%\citet{Lacasa2008} introduced a simple and fast computational method; known as visibility algorithm, to convert time series into graphs. Basically the visibility graph presents the connection between nodes in the network. For any time series, two characteristics are attributed to each event: occurrence time and value of the event. Two events are in connection or visible to each other, if no other event interrupts their linear connection. In mathematical form, event $i$ and event $j$ are visible to each other if the following equation fulfills:
%
%\begin{equation}
%\frac{y_i - y_k }{t_k - t_i} > \frac{y_i - y_j}{ t_j - t_i} ,
%\end{equation}
%
%\noindent
%where, $y$ is the value and $t$ is the time of the event. $k$ is the index of any event occurring between event $i$ and event $j$. The visibility graph generated from the time series holds the following conditions, 1) each event is visible to the first event at its right and left side (if there is any) (Connectivity), 2) The algorithm is developed without defining a direction for the links (Undirected),  and 3) Rescaling or translation of a time series is not changing the resulted visibility graph (Invariant under affine transformations of  the series data)\citep{Lacasa2008}. This concept has applications ranging in various fields including geophysical studies.
%
%\noindent
%The analysis of seismic sequence through visibility graph for various tectonic seismic regions, revealed an alternative approach to study the magnitude time series. \citet{Telesca2012} studied the seismic sequence of Italy between year 2005 and 2010 through using the visibility graph method. Applying different threshold magnitudes and observing the collapsing effect of all the distribution degrees, they argued that in analyzing the magnitude time series, VG seems to depend only on the values of the magnitude and to be independent of the threshold magnitude. In another research, \citet{Telesca2013} studied the seismicity of the Mexican subduction zone through visibility graph approach. They extracted characteristics of the visibility graph for five different tectonic seismic regions.  According to their study, visibility graph could distinguish the region with different seismicity characteristics, resulting from its different tectonic processes that governs the area. The relationship between Gutenberg-Richter parameters and $k-M$ slope (The slope of regression line fitting the Magnitude  of each event $(M)$ and its connectivity degree $(k)$) was investigated in their study. In a similar article, \citet{Telesca2014}, studied the seismicity of 2002-2011 Pannonian seismicity using the visibility graph. They classified seismic catalog into shallow and deep earthquakes and extracted the visibility graph characteristics for each class. According to their study there is a close relationship between Gutenberg-Richter $b-value$ and $k-M$ slope of the visibility graph. High linear correlation coefficient value (close to 1) of $b-value$ and $k-M$ slope suggests the universal character of the relationship between these parameters \citep{Telesca2014}.  \citet{Telesca2016} , defined the $ <T_C>$, as the parameter window mean interval connectivity time ,  as an indication of the mean linkage time between earthquakes, to study the 2003-2012 earthquake sequence in the Kachchh region of western India. They found that the $<T_C>$ changes through time, indicating that the topological properties of the earthquake network are not stationary; also $<T_C>$ significantly decrees before the largest shock of the catalog. Analyzing Gutenberg-Richter values of  magnitude time series of an earthquake sequence revealed more opportunities to study the magnitude time series using the features of networks. \\
%
%\noindent
%In this study we use visibility graph analysis to investigate the earthquake sequence of northern Iran. We divided northern Iran into three tectonic seismic regions including Azerbaijan, Alborz, and Kopeh Dagh.  Recently, northern Iran has been studied in detail from different seismological points of view. \citet{Nemati2015} studied the most recent 200 years of seismicity in northern Iran. The frequency of shocks vary widely from one main shock per 6 years (0.17 event/ year) for the Azerbaijan region to 13 earthquakes per 4 years for the Kopeh-Dagh (3.25 event/year) region \citep{Nemati2015} . \\
%
%\noindent
%Northern Iran has an elaborate seismic history.  Having considerable number of earthquakes and also being in different tectonic seismic region, northern Iran is a good platform to analyze the visibility graph method's capabilities for studying the magnitude time series. In this study we present the relationship between $k-M$ slope and the $b-value$ in three different tectonic seismic regions. We also investigate the variation of $<T_C>$ value through time.
%





