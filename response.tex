
\setlength{\paperwidth}{210mm}
\setlength{\paperheight}{297mm}% fixed.

\documentclass{article}

\usepackage[review]{myreviewpckg}

\usepackage{textcomp}
\usepackage{simplemargins}

\clubpenalty=10000  % Orphan - First of paragraph left behind
\widowpenalty=10000 % Widow  - Last of paragraph sent ahead

\setallmargins{1in}

\begin{document}

% ******************************************************************
% *************************** REVIEWER 1 ***************************
% ******************************************************************

\begin{center}
	\bf
	\large
	Authors' Response to Reviewer 1
\end{center}

\noindent
We thank the reviewer for his/her comments and suggestions, which helped us improve the manuscript. To help the review process, in the annotated manuscript version of the paper, we have highlighted the most relevant changes that resulted from these comments in green font. In the following, we provide a response to each comment. The original comments are in italic black font, followed by our responses in regular blue font.
\vspace{2ex}
\newline

\introcomment{~}{%
This is an interesting paper that applies a relatively new technique to the seismicity data in Northern Iran. The manuscript is well-written and would be a worthy contribution to the Pageoph journal. I have a few moderate-to-major concerns that I recommend to be addressed by the authors in a revised manuscript.
}

\comment{1}{%
Please explain why you decluster the catalog and be more specific on the declustering method/declustering-parameters and the results of the declustering procedure. How many events are in the catalog before/after declustering?
}

\response{%
Application of VG approach in the seismic catalog is a novel approach. Recent studies have used both declustered (e.g., Telesca et al (2014b) and Telesca et al (2016)) and the whole catalog (Telesca and Lovallo (2012) and Telesca et al (2013)). Based on this comment and subsequent comment of the reviewer (and also the comments of second reviewers), we consider both whole and declustered catalog. We used Gardner and Knopoff (1974) method for declustering the catalog and reported the catalog size for original data (whole catalog), whole catalog after considering completeness magnitude and also declustered catalog in Table 1.  
}

\comment{2}{%
In order to check how well the declustering method worked for the analyzed datasets, would be useful to show the earthquakes in a plot of longitude versus time.
}

\response{%
That is a good point. However, we think it gives an idea of distance based analysis of VG which is far from the scope of this article. We add the figure of original catalog before and after considering the completeness magnitude (whole and complete catalogs.)
}

\comment{3}{%
I think would be interesting to check/discuss/compare the k-M slope and b-value relation for both the original and the declustered catalogs.
}

\response{%
Thanks for the comment. We considered both catalogs and presented the results. Please see Fig. 7. 
}

\comment{4}{%
I would suggest to check the results for several magnitude thresholds above the completeness magnitude. Are the conclusions robust or do they depend on the chosen magnitude thresholds?
}

\response{%
The seismicity parameters change based on completeness magnitude. The parameters which come from the VG analysis also follow this pattern. According to this comment and comment \#3 of second reviewer we compute time dependent completeness test and choose a conservative completeness magnitude.  
}
 
\comment{5}{%
Would it be possible to calculate a correlation coefficient for the $k-M$ slopes and the $b-values$ in Figure 9 (middle plot)?
}

\response{%
We can add the numbers into figure 10. \\
 (Dec catalog:\ \ \ \   $ k_m :  Az =  4.4\%  Al =  7.1 \%  Ko =  6.6\% -- b: Az = 6.1\%  Al = 3.3\% Ko=4.8\%$)\\
 (Whole catalog: $ k_m :  Az =  9.4\%  Al =  7.4 \%  Ko =  6.2\% -- b: Az = 5.4\%  Al = 4.8\% Ko=5.5\%$)
}

\comment{6}{%
The authors use 20 events (in a moving window) for estimating the b-values and k-M slopes in Figure 9. I think this is an extremely small number and I am concerned about the reliability of the b-value estimates. At least for the areas of higher seismicity the authors should also present graphs obtained using a larger number of events per moving window.
}

\response{%
Thanks for the comment. We tried different window size from 20 to 100. As it is natural to expect, there are trade-offs between different window sizes. Smaller window sizes emphasize the local changes in the time series, but are less reliable when it comes to the seismicity parameters (e.g., b value). Larger window sizes, on the other hand, offer less insight into the time dependence of the window sequence characteristics. We chose a window size $n=50$, which is consistent with the suggested minimum number of events acceptable to estimate b according to  Woessner and Wiemer (2005). We modified the text accordingly. 
}

\comment{7}{%
The discussion of b-values in relation with larger earthquakes (page 10) in Figure 9 is unclear. From a quick visual inspection, I fail to see any distinct anomalies before larger (M > 5.0) events. I think that a lack of correlation should be also discussed in the manuscript, if true.
}

\response{%
The correlation for 2012 Mw 6.4 Azerbaijan earthquake is clear to see. However, we agree with the reviewer that the drops of b-value is not consistent. Or we should say in the case of big earthquake there is always a drop in b and k-m value, however, there are some drops in the window analysis which is not aligned with big earthquake. We addressed the lack of correlation for the KopehDagh region. We also discussed that the lack of correlation could be the results of relatively high completeness magnitude. The text is modified accordingly.
}


% ******************************************************************
% *************************** REVIEWER 2 ***************************
% ******************************************************************

\newpage

\begin{center}
	\bf
	\large
	Authors' Response to Reviewer 2
	\end{center}

\noindent
We thank the reviewer for his/her comments and suggestions, which helped us improve the manuscript. To help the review process, in the annotated manuscript version of the paper, we have highlighted the most relevant changes that resulted from these comments in green font. In the following, we provide a response to each comment. The original comments are in italic black font, followed by our responses in regular blue font.
\vspace{2ex}
\newline

\introcomment{~}{%
The authors present an analytical method to estimate statistical properties of earthquake catalogs and they apply their method in Iranian catalog. Although I find some promise in the method I suggest that more analysis is required for the study to be comprehensive. 
}


\comment{1}{%
1) Declustering the earthquake catalog, especially using parameters derived for California seismicity under modest monitoring conditions only brings additional uncertainty in the analysis. I am not sure why the authors believe that they should declutter the catalog. They could report if they find any time variability of b-value before/after aftershock sequences. i would suggest not going ahead with declustering.
}

\response{%
Thanks for the comment. We consider both catalogs.
}

\comment{2}{%
 Uncertainty analysis for standard b-value estimation for sample population, magnitude correction
}

\response{
Thanks. We conduct the uncertainty analysis according to Shi and Bolt 1982. Please see equation 3 and Table 1. 
}

\comment{3}{%
I find that Mc (magnitude of completeness) determination, should be done in a time-dependent manner since from the Fig. 5 it is obvious that monitoring conditions change.
}

\response{
That is right. We conduct a time dependent Mc estimation through Maximum Curvature method implemented in ZMAP software. Please see Fig. 4.  The results modified accordingly. 
}

\comment{4}{%
How the Authors account for Bam earthquake in 2003 that has aftershock in the early part of their catalog?
}

\response{
Our catalog starts from 2005. We agree the fact that  the earthquake could affect the seismicity pattern several consequent years, however, the spatial distance of the epicenter from the study region is fairly large ($d > 400 miles$). Therefore, we believe $2003 Mw 6.6$ Bam earthquake's aftershocks is not included in the current study regions' catalog.  
}


\end{document}
