
\section{Conclusion}
We studied the earthquake sequence of three tectonic seismic regions of north Iran. Results manifest the strong relationship between $k-M$ slope and Gutenberg-Richter parameters. The variation of $k-M$ slope and $b-value$ with time in sliding windows have similar behavior and drops down before big earthquakes. Significant reduction of $<T_C>$ value before big earthquake is observed. Higher coefficient of variation of $k-M$ slope in comparison with $b-value$'s CV, suggests the $k-M$ slope as better parameter for studying the magnitude time series due to higher sensitivity to window size and threshold magnitude than $b-value$. In general, $k-M$ and $b-value$ relationship preserves the behavior with variation of the window size and threshold magnitude, and could be an alternative approach for clustering the earthquake sequences based on different tectonic seismic zones.