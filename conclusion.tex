
\section{Conclusions}

We studied the seismicity of the three main seismic regions in northern Iran in the time period between January 2005 and December \myrevision{2016} using an approach based on the visibility graph method. We tested the applicability of this method for the specific region of interest and in reference to previous results from similar studies. The results confirm previous observations about the correlation that exists between the connectivity parameter $k$, the magnitude of the events in the sequence, \myrevision{the slope $m$ of the linear regression adjusted to the relationship between $k$ and $M$, and the correlation of the $b$-value from the Gutenberg-Richter law with $m$. We used the data points computed from the visibility graph analysis of the seismicity of northern Iran to obtain updated mathematical expressions for a universal relationship between $b$ and $m$, including data collected from previous studies done} for other regions as well as data from experiments. We found the relationships to have a good level of similarity, \myrevision{independently of the alternative manipulations of the seismic catalog, but noted that the results were somewhat sensitive to the completeness magnitude. Overall, our results are} indicative of the general nature of the relationship between $m$ and $b$. We also explored the relationships \myrevision{between the visibility graph properties and the seismicity parameters for a random selection of earthquake subsequences and for a progression of subsequences in time. In these cases, we found the relationship between $b$ and $m$ to hold for subsequences with a sufficiently large number of events ($n \geq 200$), but noticed they are in better agreement with the universal regressions when larger, entire regional sequences are considered. For the particular regions considered in this study, we were not able to detect conclusive evidence of other relationships between the visibility graph properties and the temporal occurrence of large earthquakes, as it has been suggested in previous similar studies. We believe this is due to the lack of larger magnitude events ($M_w > 6$) or relatively high values of $M_c$ in the selected catalogs.}

\myrevision{While the connection between the topological properties of the visibility graph of an earthquake sequence and the seismicity parameters of a region is an attractive finding, it remains to be seen how this can be put to use in a more practical scientific sense. We believe future efforts in understanding the mathematical nature of the observed relationships is a natural first step to follow. It will also be necessary to test the relationships in additional seismic regions with uncharacteristic $b$ values, especially in the ranges outside the typical $b$-values (0.8--1.2). It would also be of interest to build visibility graphs based on other earthquake parameters (e.g., moment) or considering directionality, and investigate other relationships. At this point, nonetheless, the method as used here} seems to provide an alternative and interesting approach to the analysis of the seismicity of a region.

% OLD MATERIAL
% -----------------------------------------------------------------

% that can be drawn from a moving window visibility graph analysis and the variation of the seismicity in the region of interest in time. \myrevision{We found that the }

% \myrevision{The relationship found between k-M slope and b-value suggest that the topological properties of the VG are not unlinked with the seismological properties of  an earthquake sequence.} 

% We found there may be a potential relationship between the $b$ value and the occurrence of earthquakes as well as with the graph's mean interval connectivity time parameter <$T_c$>, but additional research in regions with stronger events may be necessary before drawing stronger conclusions in this regard. \myrevision{Also further investigation has to be performed in the future to clarify the relationship between the k-M slopes as a pure mathematical object and the b-value and investigate the k-M slope with other earthquake network models.} The method used here, nonetheless, seems to provide an alternative and interesting approach to the analysis of the seismicity of a region. 

% We studied the earthquake sequence of three tectonic seismic regions of northern Iran. Results manifest the strong relationship between  $k-M$  slope and Gutenberg-Richter parameters. The variations of  $k-M$  slope and  b-value  with time in sliding windows have similar behavior and decline before large earthquakes. A significant reduction of  $<T_c>$  value before large earthquakes was observed. The higher coefficient of variation of  $k-M$  slope in comparison with  b-value  suggests that the  $k-M$  slope is a better parameter for studying the magnitude time series due to higher sensitivity to window size and threshold magnitude than  is found in the $b-value$. Sensitivity analysis revealed that the VG parameters behave independently than threshold magnitude and number of events. In general,  the $k-M$  and  $b-value$  relationship preserves the behavior with the variation in the window size and threshold magnitude, and could be an alternative approach for clustering earthquake sequences based on different tectonic seismic zones.