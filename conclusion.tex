
\section{Conclusion}
\noindent
We studied the earthquake sequence of three tectonic seismic regions of northern Iran. Results manifest the strong relationship between  $k-M$  slope and Gutenberg-Richter parameters. The variations of  $k-M$  slope and  b-value  with time in sliding windows have similar behavior and decline before large earthquakes. A significant reduction of  $<T_C>$  value before large earthquakes was observed. The higher coefficient of variation of  $k-M$  slope in comparison with  b-value  suggests that the  $k-M$  slope is a better parameter for studying the magnitude time series due to higher sensitivity to window size and threshold magnitude than  is found in the $b-value$. In general,  the $k-M$  and  $b-value$  relationship preserves the behavior with the variation in the window size and threshold magnitude, and could be an alternative approach for clustering earthquake sequences based on different tectonic seismic zones.