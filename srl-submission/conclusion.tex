
\section{Conclusions}

We studied the seismicity of the three main seismic regions in northern Iran in the time period between January 2005 and December 2015 using an approach based on the visibility graph method. We tested the applicability of this method for the specific region of interest and in reference to previous results from similar studies. The results confirm previous observations about the correlation that exists between the connectivity parameter $k$, the magnitude of the events in the sequence, the slope of this relationship or $k$-$M$ slope, and the $b$-value from the Gutenberg-Richter law. We obtained mathematical expressions for the region of interest as well as for so-called universal data collected from this study and previous additional analysis for other regions as well as data from experiments and found the relationships to have a good level of similarity. This is indicative of the general nature of the relationship between the $k$-$M$ slope and the $b$ value. We also explored the potential relationships that can be drawn from a time-dependent visibility graph analysis and the variation of the seismicity in the region of interest. We found there may be a potential relationship between the the $b$ value and the occurrence of earthquakes as well as with the graph's mean interval connectivity time parameter <$T_c$>, but additional research in regions with stronger events may be necessary before drawing stronger conclusions in this regard. The method used here, nonetheless, seems to provide an alternative and interesting approach to the analysis of the seismicity of a region.

